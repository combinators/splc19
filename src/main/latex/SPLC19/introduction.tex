%% 
%% Introduction
%%
%% Key ideas: 
%%
%% 1. Augmenting feature models with application domain modeling
%% 2. Impact this has on the toolsuite for featureIDE and composition engines
%%
%% 3. Expose programming API instead of providing a standard black-box
%%    capability, which might otherwise restrict users' abilities.

Software product lines (SPLs) refer to software engineering methods,
tools and techniques for creating a collection of similar software
systems, called product line members, from a shared set of software
assets using a common means of production. A number of characteristics
separate SPLs from routine large, complex systems and thus there is a
need for specialized tools and techniques to design, implement and
maintain SPLs.

Fundamental to many SPL approaches is the concept of a \textit{Feature
Model}, first introduced by the Feature-Oriented Domain Analysis (FODA)
technique~\cite{Kang1990} and refined over the years by numerous
researchers. A feature model is composed of user-visible features from
which a wide variety of configurations can be defined through the
selection of individual features. Many tool chains have been developed
to generate product line members ``on demand'' from a feature model
configuration. Developers need to construct the SPL code artifacts in a
specific way that enables the tools to work; this may include, for
example, low-level compiler directives embedded in source code, or
specialized languages (such as JAK~\cite{Batory2004FeatureorientedPA} or
DeltaJ~\cite{Schaefer:2010:DPS:1885639.1885647}) that developers use to
encode the artifacts.

%% ,natwidth=1154,natheight=576
%%\begin{figure}
%%  \includegraphics[scale=0.25]{gpl2.png}
%%  \caption{Graph Product Line Feature Model}
%%  \label{fig:feature-model}
%%\end{figure}

There remains a potential issue, however, when the variability inherent
in the product line relates to the application domain model. The common
result is that one must integrate these domain-model concepts into the
feature model so they can become part of the configuration process. The
feature model from Figure~\ref{fig:feature-model} contains features
relating to behavior (i.e., \textbf{Prim} contains Prim's implementation
of the Minimum Spanning Tree problem) and some relate to structure
(i.e., \textbf{Weighted} captures the notion that edges in a graph may
have weights), while a third set (i.e., \textbf{KruskalPrepGnR}) are
features that mediate between the behavior and structure. As one might
expect, since a Feature Model is global in scope (and uniformly
decomposed into features), the dominant decomposition can fail to
account for the interrelationships of the features.

In past work, we have explored product lines within a number of
application domains~\cite{Heineman:2015:TMO:2791060.2791076,PEPM18} as
we investigate the role that domain modeling should play within an SPLC.
In particular, given the extensive developments in model-driven
engineering, we looked into ways to integrate a separate domain model
into the tool chain used for generating product line instances. We
demonstrate how to incorporate domain modeling within the Graph Product
Line, a canonical SPLC used extensively within the research community.
Our focus is on using code generation to automatically generate the
boilerplate code that would otherwise have to be customized based on
structural considerations determined by the domain model. Our aim is to
make it easier to design the code artifacts for individual features that
will be able to compose together regardless of the structure of the
domain model. In a way, we are inspired by the underlying philosophy of
the Demeter project~\cite{karl:demeter} which aims to separate objects
from their operations, to allow each to evolve independently without
significant impact on each other.

%We make the following observations regarding product lines: % %--
%--Should be able to add to and remove features from a product line
 %after it has been defined  %--A product line definition is still
 %reflected in software artifacts, which means it should support common
 %refactoring functionality % %--Should be feature-rich, which means one
 %can potentially envision a significant number of PL members (e.g., not
 %just a handful) % %Without one of these characteristics, it could only
 %be considered a closed system.


% A SPL is a family of related programs. When the units of program
% construction are feature—increments in program functionality or
%development—every program in an SPL is identified by a unique and legal
%combination of features, and vice versa. Family member refers to
%individual product. A variation point represents a decision leading to
%different variants of an asset. A variation point consists of a set of
%possible instantiations (legal variations of the variation point). A
%variation point usually specifies the binding times, that is the
%time/times at which a decision about the instantiation has to be taken.

%The variant derivation is the action in which assets are combined from
%the set of available assets and contained variation points are
%bound/instantiated. If there are variation points with multiple binding
%times, the derivation will happen stepwise at each binding time. The
%result of such a derivation is a set of derived assets. The derivation
%can be executed technically in many ways. The simplest way is to copy
%assets and modify (parts of) them (e.g. the configuration parameters)
%manually. The result of such a derivation is often called a
%configuration.



%% THESE NEED TO BE PLACED IN CONTEXT. Here is a reference to
%K\"{a}stner's paper~\cite{Kastner:2012}. % %Here is a reference to
%~\cite{leich2005tool}. Here is a reference to %~\cite{proksch2014tool}.
%Here is a reference to ~\cite{7203038}. Here is %a reference to
%~\cite{Sayre:2005:UMA:1082983.1083277}. Here is a %reference to
%~\cite{Setyautami:2016:UPD:2934466.2934479}. Here is a %reference to
%~\cite{Sousa:2016:EFM:2934466.2934475}. % %Here is a reference to
%~\cite{Arcaini:2017:ARV:3106195.3106206}. % %Here is a reference to
%~\cite{Vasilevskiy:2016:TRP:2934466.2934484}. % %Here is a reference to
%~\cite{Kuhn:2015:CPC:2791060.2791092}. % %Here is a reference to
%~\cite{Schaefer:2010:DPS:1885639.1885647}. % %Here is a reference to
%~\cite{Apel:2008:AFF:1428476.1428480}.


\subsection{Approaches to Product Line Development}

Software product lines create unique engineering challenges for a number
of reasons. First, it should be possible to generate any number of
instances of the product line, and by this we mean the code artifacts
for the instance application. These are then compiled (or interpreted)
to realize the execution of the application. Second, there are multiple
development efforts; one can work on code that will effectively be used
by all product line instances, but at other times, one is focused on
writing code for just a single instance from the PL.

A major goal of featured-oriented PL is to derive a product
automatically based on user’s selection. There are three philosophical
approaches widely used in practice -- \textit{annotation-based} ,
\textit{composition-based} or \textit{component-based} -- which differ
in the way they represent variability in the code base and the tool
chains used to generate the product line members.

\subsubsection{Annotation-based Approach}

An annotation-based SPLC consists of a single body of code artifacts
which fully contains all code resources used by all members of the
product line. Using different tools or language-specific capabilities, a
compiler (or pre-processor) extracts subsets of the code to be used for
a PL instance. One of the most common approaches is to use compiler
directives embedded within the code as a means for isolating code unique
to a subset of product line instances. Then each product line instance
can be generated by compiling the same code base with different compiler
flags, resulting in different executable instances. Due to the nature of
this approach, often one cannot review the source code for individual
instances~\cite{Apel:2013:FSP:2541773}. Often the compiler directives
manipulate the core data structures or class definitions of the
software, for example, adding or removing a class field
definition~\cite{Liebig:2010:AVF:1806799.1806819}.

Annotation-based approaches are widely used in practice because they are
easy to use and are already natively supported by many programming
environments. While these approaches are easily adopted, the limited
(and typically non-extensible) tool support can make it more complicated
to maintain an SPLC in the face of constant evolution and added (or
removed) features. A product-line-aware type checking
system has also been developed to statically and efficiently detects
type errors in annotation-based product-line.
implementations~\cite{Kastner:2012,Apel:2013:FSP:2541773}.

\subsubsection{Composition-based Approach}

Another common approach is to design a \textit{feature model tree} to capture the
externally visible features that differentiate one product line instance
from another. Each feature can include any number of associated code
artifacts. A product line instance is configured by selecting
for inclusion features from the feature tree, potentially restricted by
constraints. A composition engine processes the code assets associated
with the selected features to create the final source code for the
product line instance.

Composition-based approaches locate code belonging to a feature or
feature combination in a dedicated file, container, or module. A classic
example is a framework that can be extended with plug-ins, ideally one
plug-in per feature. The key challenge of composition-based approaches
is to keep the mapping between features and composition units simple and
tractable.

Another way to view the difference between annotation and composition is
that annotation separate concerns virtually and composition separate
concerns physically,and code is removed on demand with annotation while
composition units are added on demand.
~\cite{Thum:2014:FEF:2537169.2537315,Apel:2013:FSP:2541773}.

\subsubsection{Component-based Approach}

A software component is a unit of composition with contractually
specified interfaces and explicit context dependencies only. A software
component can be deployed independently and is subject to composition by
third parties. The key idea of a component is to form a modular,
reusable unit.

Developing product line by constructing and composing reusable was a
common strategy. With domain analysis, developers decided which
functionality should be reused across multiple products of the product
line designed components accordingly. To derive a product for a given
feature selection during application engineering, a developer selects
suitable components and then manually writes glue code to connect
components for every product individually.



\subsubsection{Product Line Techniques In Industry}

Annotation-based approach are not so widely used as composition-based
approach because drawbacks mentioned above. Here are some existing
annotation-based approaches:

CIDE~\cite{CIDE:Eclipse} is an Eclipse plug-in that
replaces the Java editor in SPL projects. Developers start with a
standard Java legacy application, then they select code fragment and
associate them with features from the context menu. The marked code is
then permanently highlighted in the editor using a background color
associated with the feature. Researchers have shown how to type check
software product lines to identify and eliminate common errors, such as
missing class fields or methods.

Boost is a set of libraries for the C++ programming language that provide support for
tasks and structures.


To work with FeatureIDE, the primary challenge is to design a feature
tree model to represent the desired product line application domain.
Because features are cross-cutting with regards to the artifacts in the
programming language, the various composer engines supported by
FeatureIDE accomplish the same goal in a variety of ways.

AHEAD has feature modules for each concrete feature, and the
corresponding composition tool places generated source code directly
into the Eclipse source folder. AHEAD brings separate tools together and
selects different tools for different kinds of files during feature
composition,establishing a clear interface to the build system.
Composing Jak files will invoke a Jak-composition, whereas composing XML
files invokes an XML-composition
tool.~\cite{Batory2004FeatureorientedPA}.

FeatureHouse tool suite has been developed that allows programmers to
enhance given languages rapidly with support for feature-oriented
programming. It is a framework for software composition supported by a
corresponding tool chain. It provides facilities for feature composition
based on a language-independent model and tool chain for software
artifact, and a plug-in mechanism for the integration of new artifact
languages.~\cite{Apel:2009:FLA:1555001.1555038}

Deltaj is a Java-like language which allows to organize classes in
modules. A program consists of a base module and a set of delta module
in a stepwise manner. Much like a feature module, a delta module can add
new classes and members as well as extend existing methods by
overriding. In contrast to feature modules, delta modules can also
delete existing classes and individual
members.~\cite{Schaefer:2010:DPS:1885639.1885647}.

In LaunchPad, each feature can contain any number of combinators,
designed using a DSL we had developed to simplify the writing of
combinators for an earlier CLS tool. A configuration is a subset of
features from which a repository is constructed. Each feature can
optionally store target definitions, which are aggregated together and
then used as the basis for the inhabitation requests, i.e.As each
request is satisfied, the synthesized code from the resulting type
expression M is stored in the designated source folder.
~\cite{Heineman:2015:TMO:2791060.2791076}.

\subsubsection{Evaluation of related work}

The key challenge of composition-based approaches is to keep the mapping
between features and composition units simple and tractable.
Preprocessor-based and parameter-based implementations are often
criticized for their potential complexity, lack of modularity, and
reduced readability. And they all have some problems which is the
difficulty to refactoring.~\cite{Kim:2017:RJS:3106195.3106201}.

Although component-based implementations are common in product line
practice, they lack the automation potential of feature orientation that
we aim at. Deciding when to build a reusable component and what to
include in that component is a difficult design decision, one need to
have good understanding of whole scope to do that. There is no domain in
this approach, glue code is a necessary, it's no more than assembling
components. ~\cite{Apel:2013:FSP:2541773}

In practice most PLs appear “from the ground up” where developers take
advantage of language-specific capabilities to annotate different code
regions as being enabled (or disabled) based on compiler directives.
Starting from an annotation-based code repository many composition-based
approaches simply “snipped” or refactored code fragments to recreate
countless tiny “features” that could be selected.

Manual composition is a configuration process. A designer selects
individual features from a feature model and relies on constraints to
ensure the resulting product line member is valid. Manual Composition is
limited to a potential total of 2N configurations where N represents the
number of available features in the model. There is no domain modeling,
What commonly occurs is the designer must make sure that changes to any
of the units will not invalidate those product line members that
incorporate that feature.

Another problem is that the features are fixed and unchanging. If we
need to make some modifications to current instances, we may need to
trace all the way back and change the code in many classes because it’s
inheritance structure. If we want to add features which is slightly
different from existing ones, we may need to start from very beginning.

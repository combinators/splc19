% New Approach % As we investigated the way in which application domains
were represented in feature models, we identified a number of
limitations in how modeling was represented and devised a number of
strategies for improving the way these application domain models could
improve the process of generating product line members.

There are features to represent the structure of a variations, and there
is a feature for each variation. Our goal was to support the easy
construction of new variations by reusing existing features where
possible and adding new features as needed to support the functionality
expected of new variations.

\subsection{Integrated application domain models}

The concept of an application domain model is common in software
engineering because of its ability to identify real-world concepts that
will ultimately be present in a software system. When the model is
actively maintained over time, as changes are introduced and the
software system evolves, there is the opportunity to take advantage of
its high-quality information.

Annotation-based approaches for SPLs that rely solely on the embedded
compiler directives within the code are incompatible with a separately
designed application domain model. When product lines use compiler
directives to alter the very structure of the object-oriented classes in
the code base, as observed in~\cite{Kastner:2012}, there needs to be
extensive type checking to ensure only valid product line members are
generated. In these cases, one typically finds that ``the code is the
design'' and there no other separate representation of the classes being
manipulated by the compiler directives. For this reason, our
investigation focused primarily on composition-based SPLC approaches.

Researchers have investigated various object-oriented technologies to
support SPLs~\cite{Griss1999,Batory2000} with the focus of finding ways
to develop reusable components within an SPLC, whether using variations
of \textit{mixins}~\cite{Bracha:1990:MI:97946.97982} or C++
Templates~\cite{VanHilst:1996:UCT:646898.710025}. What is missing,
however, is the capability for the application domain model to be fully
integrated into the SPLC tool chain and have as great an influence as
the features themselves. We observed in nearly every approach using
feature models that the lack of a domain model resulted in increasingly
complicated feature models. Specifically, numerous sub-features appeared
to be nothing more than instantiations of different configurations, as
we first identified in Figure~\ref{fig:feature-model}. This is most
evident when reviewing sample product lines as supported by
FeatureIDE~\ref{Thum:2014:FEF:2537169.2537315}, an extensible
Eclipse-based framework for feature-oriented software development. A
product line member is defined solely by the configuration of existing
features in the feature model, as allowed by the defined constraints.
Because there is no separate domain model, the various FeatureIDE
approaches all appear to have configurations which become ineffective
domain modeling. For example, in some FeatureIDE models, there are
features with sub-features that appear to be nothing more than
instantiations of different configurations, which does little to reduce
the overall complexity, and instead, simply widens the feature model tree.

\subsection{CLS generic composition}

With dominated approach for using feature in PLs, n features in the
feature tree may generate 2n configurations which will become product
line instances. But if we use CLS as the algorithm for composition, the
fundamental units will be combinators instead of features. The CLS
starts with a repository of combinators to which a user issues a query
which attempts to find a type in the repository using inferencing. Using
CLS generic composition, limitations of composition tools are
eliminated.


Combinators can be dynamic and added at composition time, something
which is simply not possible in traditional feature trees used by
feature-oriented product lines.

To better explain these dynamic combinators, consider having a feature
model with a feature that provides variability and there are a number of
fixed sub-features that are tailored for each valid variation. For
example, “Number of external hard disks” might have sub-features
“One-Hard-Disk”, “Two-hard-disks” and so on. Individual members of the
PL are configured, accordingly, to select the desired number of external
hard disks. In contrast, using CLS a single combinator class
NumberOfExternalDisks is parameterized with an integer, and one can
instantiate a combinator (NumberOfExternalDisks(3)) and add to the
repository as needed based on the modeling needs of the member.

Without making 2n configurations, using CLS will significantly simplify
code system in PL, optimize code structure make it more readable and
reasonable.


\subsubsection{Compositional manipulation}

Feature-IDE relies on externally provided composition engines to process
code fragments. The challenge is that FeatureIDE can make no semantic
guarantees about the resulting code. Also there is no theoretical
foundation for the composition, which rather simply is assembly. During
assembly, units are wired together without making any changes to the
units themselves.

\subsubsection{Language agnostic}


Without being language limited as normal ways, our approach is more
laguage agnostic. Choice of language have been more diversified, which
could benefit more engineers with different backgrounds. We don't have
to take advantage of single language to build our code base. For
example,we have to use .jak files in AHEAD. If you are not familiar with
the language, you can't use the approach.~\cite{PEPM18}.

\subsubsection{Code sharing between assets}


Like we mentioned above, dynamic combinators can be constructed to add
methods into classes. Assets can share some basic code, with different
methods included.

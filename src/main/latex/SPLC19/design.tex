


Our improved solution successfully integrates application domain modeling with CLS to dramatically improve our
ability to automatically compose members of a product line structure. We start by designing a domain model.
This domain model simply models aspects of these variations, and define constraints between assets. For example,
MST algorithm can only be applied on undirected and weighted graph. Then we construct concrete combinators and dynamic
combinators to represent instances, which are added to gamma repository. After targets are identified, the next step
is to inhabit.~\cite{Ddder2013UsingII}. We instantiate our combinators, perform reflection on them to allow the CLS engine to gather their type
information, and then ask CLS to produce all compositions (by invoking apply methods).~\cite{Heineman:2015:TMO:2791060.2791076}.
Each valid configuration is an instance of the model.

We have achieved the following vision: to introduce a new member of the product line or to modify an
existing one –just create (or modify) the application domain model and synthesize the member using
automatic composition. When a variation includes unique logic that has not been seen before, then appropriate
 localized Scala code changes are introduced; if multiple variations could benefit from the variation-specific concepts,
  then the Scala code is refactored to bring this logic into the shared area. We combine the advantages
  of feature model and object-oriented model together and discard  the limitations.
  The light-weight partial modeling allows us to model a small part we are developing
 ,make the domain extensible and flexible. Code synthesis can generate all boilerplate code directly
 that we can focus more on the inner logic of the system, there will be no necessary that we
 copy paste large amount of code like what people did in AHEAD.

 \subsection{More powerful Scala Implementation}

 By implementing CLS in a general purpose language, we move away from an interpreted DSL towards an embedded DSL
 that increases expressivity. However, by choosing Scala we gained a number of significant benefits as well:

 • Scala program immediately interoperate with numerous Java libraries in the wider software ecosystem. Notably,
 we integrate JavaParser to manipulate ASTs and manage the synthesized software using jGit to
 provide API-level access to the Git version control system.
 • Scala offers expressive string interpolation to embed variable references and entire source code fragments within strings.
 We use this extensively to make it easy for programmers to write readable combinators.
 • IntelliJ offers a powerful Scala environment that seamlessly integrates the CLS tool, so programming with
 combinators is treated the same as programming in a native language. As code is synthesized, it is placed
 into local Git repositories and then checked out into IntelliJ.

 In our approach, we consturct two kinds of combinators with CLS technology. We have extensible static @combinator for all base classes,
  like Vertex and Graph, which is necessary for the whole product line. Once you add them into repository,
   they will always be there. Listing below shows few line of vertexBase which synthesize the base class for a vertex:

\begin{lstlisting}{language=scala}
@combinator object vertexBase{
def apply(extensions:Seq[BodyDeclaration[\_]]):
   CompilationUnit = { Java(s"""
      |public class Vertex  {
      |${extensions.mkString("\n")}
      |}""".stripMargin).compilationUnit
}
val semanticType: Type =
  vertexLogic(base, extensions)
  =>: vertexLogic(base, complete)
}
\end{lstlisting}

The apply method has a parameter which will ultimately be resolved by CLS. It takes a class gives
Seq[BodyDeclaration[\_]] with vertexLogic(vertexLogic.base,vertexLogic.extensions) as semanticType. A combinator is instantiated
from this class–based on a specific application domain object that models the desired variation and added to gamma
 repository. \verb+${extensions.mkString("\n")}+ is where the extra imports, fields, and methods required for
 the specific variation determined by domain object will be added later by dynamic combinator. The types of the parameters designed by semanticType ensure
 that the arguments are all appropriate, and so the resulting synthesized class will have no syntax errors.

Instead of relying solely on annotated fixed combinators (as shown in Listing above), we can programmatically
add customized dynamic combinators(as shown in Listing below). This ability was a major step forward in our ability to populate with modular
units suitable to compose any number of instances from a product line structure. For example ,we want to add color
attribute to vertex:

\begin{lstlisting}{language=Scala}
class ColoredVertex {
  def apply() : Seq[BodyDeclaration[\_]] = {
    Java(s"""
     |    private int color;
     |
     |    public void setColor(int c) {
     |        this.color = c;
     |    }
     |    public int getColor() {
     |        return this.color;
     |    }
     |""".stripMargin).classBodyDeclarations()
 }
 val semanticType: Type = vertexLogic(base, var_extensions)
}
\end{lstlisting}

By using Scala rather than a DSL we can immediately take advantage of writing reusable and extensible combinators.
In AHEAD approach, there are many duplicated code, for example, the refinements to Edge are exactly same in MSTKruskalPrepGr
 and MSTPRIMPREPGnr. Also some empty methods are there only for future override. If you do not have a solid understanding
  for the whole scope in the beginning, which happens a lot in industries, it's hard to construct a structure like that.
  With our approach, it's much easier to construct product lines from scratch, and the code will make more sense.
  In Ahead approach, some features can only be used for once. For example, only thing feature dProgStronglyConnected do
  was to provide a run method. This will lead to redundant code, while our approach doesn't have such problems.

\subsection{More flexible code construction}

 Other than adding content to existing class, which can also be realized with AHEAD by use refinement in JAK file,
 we can also modify or even delete content in place easily by constructing a combinator which takes a CompliationUnit
  as parameter, customize the class and return it as a CompliationUnit.

  %%Code example will be added here

 With our approach, it makes it possible to customize existing classes by either adding content or just changing it.

\subsection{BoilerPlate ready to go}

Boilerplate code can be wrote once and applied in any number of contests. In widely used approach Boost,
you have to be an expert in C++ and the Boost library to use the API. With our approach, all the complicated code could
be synthesized, programmers can focus on designing artifacts.

For example, in Prim algorithm, we need couple loops with iterator to go through the list of vertices and
 edges. While in our approach, we can add content between these two interfaces
 ${gen.getVerticesBegin("vx")} and ${gen.getVerticesEnd("vx")} to go through the list and use
 information of the vertices.

here we ~\cite{Schwagerl:2016:STS:2970276.2970288}.

 \subsection{Perform Computations During Synthesis}

Once the combinators (both static and programmable) are added to repository, the inhabitation targets are requested
and a forest of type expressions is computed that satisfy these requests. To complete the code generation, the apply
methods of the respective combinators in the types are computed.


 \subsection{Map from Application to Solution Domain}

 Our most successful contribution is the ability to create customizable and extensible code generators that produce
 solution domain code fragments directly from application domain elements.
 This capability truly shows how to transform a lightweight application domain model into executable solution domain
 code, without relying on proprietary or complicated tool support. The resulting product line members are constructed
 more directly than existing configuration mechanisms. Code synthesize is more like a process of assembling
 model.

 The automated composition technology reveals the direct links between the application domain and the solution
 domain while allowing each to be independently evolved and modified. Evolution can occur in the application domain
 (i.e., constraints can be added or existing ones refactored) or the solution domain (i.e., an API for a framework
 changes and the corresponding invocations must adapt). Because there is no design tool separating the programmer
 from seeing and modifying the combinators, our approach provides maximum power and flexibility in designing and
 integrating variation speific combinators. Note that the code generator registries are an independent contribution
 of this paper, and their behavior is orthogonal to the CLS algorithm.

 \subsection{Engineering Product lines}

 The CLS algorithm is invoked as a locally hosted web service executed from within IntelliJ as a background task.
 As combinators are designed and constructed, IntelliJ helps programmers with all of the modern conveniences
 provided by an IDE, such as code completion, syntax hilighting, and searching (both definitions and usage of).
 Each graph variation is configured using a top-level Scala class, such as shown above.
 To request construction, visit localhost:9000/simple in a web browser and the designated code executes.

 As customized by the application domain model, a repository of static combinators is constructed
 based on the Scala traits that contain variation-specific combinator definitions.And all programmable combinators
 are added to the repository and, using reflection, the CLS engine gathers all type information.
 There are two steps to synthesize a specific graph variation. First, upon activation, the CLS web-service
 performs type inhabitation on the requests which are derived from the application domain model instance or variation.
  These targets are programmatically supplied, rather than hard-coded.
  The result of inhabitation is a forest of type expressions returned to
 the client invoking the web-service. Second, instead of simply pretty printing the synthesized code
 , the web-service packages the synthesized code corresponding to the type
 expressions in a locally-hosted Git repository.

 Since all artifacts are programmed natively, we can immediately increase programmer
 productivity in at least four ways; (1) use breakpoints in IntelliJ within the combinator Scala code to debug
 step-by-step through the static or programmatic combinators; (2) use standardized logging capabilities
 (with different warn, debug, and error levels) to generate log files to record the detailed context when debugging
 complicated combina- tors; (3) incorporate unit tests for individual combinators as well as the POJOs used to
 model the application domain, so programmers can increase their confidence in the modular units; (4) use code
 coverage analysis over the test cases to identify non-executing code and potential weak areas;

 Using CLS is a major improvement over burying build details within Makefile/Ant scripts which encode the detailed
 knowledge for constructing individual product line members. Since the entire composition process is transparent,
 programmers have full control over the process; for example, we have scripts to automatically synthesize and
 compile all registered solitaire variations.
 A limitation of FeatureIDE is that the user must manually choose the current configuration, and only one composed
 product line member can exist at a time. Finally, the CLS composed code is readable, and at first glance is
 indistinguishable from manually written software.
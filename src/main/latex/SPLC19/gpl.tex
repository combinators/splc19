%% Here is where to describe GPL


The Graph Product Line (GPL) is a well-known case study within the software product line community
~\cite{Lopez-Herrejon:2001:SPE:645418.652082}.Graph is a fundamental topic in computer science and choosing it minimises the requirement
of becoming a domain-expert which is a prerequisite in the product line development process. Still it is complex
enough to expose the underlying concepts of product lines and their implementation. This product line
 supports variations in a library of graph data strutures and algorithms. A possible feature diagram of the graph
 library is shown in Figure?.Like any other product line, members of the GPL share some common features and differ in
 certain variant features. The root is labeled with Gpl to represent a graph product. It has a mandatory child feature
 GraphType, because each graph library has to implement an type, which is either Directed or Undirected. Furthermore,
 three other child features of the root are optional: Search, Weighted and Algorithm. Search strategies may be either
  breadth-first search (BFS) or depth-first search (DFS). Algorithm offers a selection of graph algorithms as child
  features. Since it's optional, either zero, one, ore more algorithms may be presented in a graph product. In our
  example, the algorithm for minimal spanning trees MST has two alternative implementations, Prim and Kruskal. Some
  non-local conditions are modeled as explicit Boolean constraints-- for example, minimal spanning tree make only
  sense for weighted graphs, and shortest paths can be computed directed graphs only.

  Some of the common and variant features that the GPL members share are identified as
  follows:

  • Common Features: The common features that the GPL applications share are Vertex and Edge.
  Which means a GPL application must have these features.

  • Variant Features: Members across the GPL share some variant features:

  – Graph Type: A graph is either directed or undirected.
  – Weight: Edges in a graph can be weighted with non-negative numbers or unweighted.
  – Search: A graph application can implement at most one searching algorithm, Breadth First
  search (BFS), Depth First Search (DFS) or none.
  – Algorithms: A graph application implements one or more of the following algorithms: Number
  (Vertex Numbering), Connected (Connected Components), Strongly (Strongly Connected
  Components), Cycle (Cycle Checking), Shortest (Single-Source Shortest Path) and MST(Minimum
  Spanning Tree).Details of these algorithms can be found in any algorithm book and more information
  on GPL can be found at: http://www.cs.utexas.edu/users/dsb/GPL/graph.htm.
  AND http://stg-tud.github.io/sedc/Lecture/ws16-17/6-SPL.pdf.

Here are papers using GPL ~\cite{Thum:2011:AFF:2061045.2062153}.
~\cite{ZhangGraphPL}.
~\cite{LopezHerrejon2014TowardsAB}.
~\cite{Bagheri:2010:CSP:1885639.1885642}.
~\cite{Lopez-Herrejon:2012:TFI:2110147.2110158}.
~\cite{Johansen2010ExploringTS}.
~\cite{doi:10.1002/smr.534}.
~\cite{10.1007/978-3-642-31095-9_40}.


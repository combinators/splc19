%
% The first command in your LaTeX source must be the \documentclass command.
\documentclass[sigconf]{acmart}

\usepackage{listings}

%
% defining the \BibTeX command - from Oren Patashnik's original BibTeX documentation.
\def\BibTeX{{\rm B\kern-.05em{\sc i\kern-.025em b}\kern-.08emT\kern-.1667em\lower.7ex\hbox{E}\kern-.125emX}}
    
% Rights management information. 
% This information is sent to you when you complete the rights form.
% These commands have SAMPLE values in them; it is your responsibility as an author to replace
% the commands and values with those provided to you when you complete the rights form.
%
% These commands are for a PROCEEDINGS abstract or paper.
\copyrightyear{2018}
\acmYear{2018}
\setcopyright{acmlicensed}

\acmConference[SPLC'19]{23rd International Systems and Software Product Line Conference}{9--13 September, 2019}{Paris, France}
\acmBooktitle{Woodstock '18: ACM Symposium on Neural Gaze Detection, June 03--05, 2018, Woodstock, NY}
\acmPrice{15.00}
\acmDOI{10.1145/1122445.1122456}
\acmISBN{978-1-4503-9999-9/18/06}

%
% These commands are for a JOURNAL article.
%\setcopyright{acmcopyright}
%\acmJournal{TOG}
%\acmYear{2018}\acmVolume{37}\acmNumber{4}\acmArticle{111}\acmMonth{8}
%\acmDOI{10.1145/1122445.1122456}

%
% Submission ID. 
% Use this when submitting an article to a sponsored event. You'll receive a unique submission ID from the organizers
% of the event, and this ID should be used as the parameter to this command.
%\acmSubmissionID{123-A56-BU3}

%
% The majority of ACM publications use numbered citations and references. If you are preparing content for an event
% sponsored by ACM SIGGRAPH, you must use the "author year" style of citations and references. Uncommenting
% the next command will enable that style.
%\citestyle{acmauthoryear}

%
% end of the preamble, start of the body of the document source.
\begin{document}

%
% The "title" command has an optional parameter, allowing the author to define a "short title" to be used in page headers.
\title{Towards Automated Composition of a Product Line}

%
% The "author" command and its associated commands are used to define the authors and their affiliations.
% Of note is the shared affiliation of the first two authors, and the "authornote" and "authornotemark" commands
% used to denote shared contribution to the research.
\author{Shengmei Liu}
\email{sliu7@wpi.edu}
\affiliation{%
  \institution{WPI}
  \streetaddress{100 Institute Road}
  \city{Worcester}
  \state{Massachusetts}
  \postcode{01609}
}

\author{George T. Heineman}
\email{heineman@wpi.edu}
\affiliation{%
  \institution{WPI}
  \streetaddress{100 Institute Road}
  \city{Worcester}
  \state{Massachusetts}
  \postcode{01609}
}

%
% By default, the full list of authors will be used in the page headers. Often, this list is too long, and will overlap
% other information printed in the page headers. This command allows the author to define a more concise list
% of authors' names for this purpose.
% \renewcommand{\shortauthors}{Trovato and Tobin, et al.}

%
% The abstract is a short summary of the work to be presented in the article.
\begin{abstract}
Object-oriented (OO) frameworks represent a significant achievement in extensible design,
but there are many well-documented challenges when third-party programmers attempt to use and
refactor them. In earlier work, we described how to migrate existing OO framework- based software
into a software product line structure using combinatory logic synthesis (CLS) integrated into FeatureIDE,
an Eclipse-based IDE that supports feature-oriented software development. While initially successful at
synthesizing a few instances of a product line, the approach does not scale to support larger product
lines because it does not adequately capture the commonality and inherent variability in the application domain.
In this paper, we analyze these problems, and introduce our new approach to construct product line which
overcomes many drawbacks of old approach. We describe how application domain modeling helps automated
composition, and how our redesigned CLS-engine improves scalability. Results are illustrated by scaling
the well-known graph product line with our approach, while reducing the average amount of instance-specific
code significantly, generating more readable code and providing more convenience for refactoring.
\end{abstract}

%
% The code below is generated by the tool at http://dl.acm.org/ccs.cfm.
% Please copy and paste the code instead of the example below.
%

%
% Keywords. The author(s) should pick words that accurately describe the work being
% presented. Separate the keywords with commas.
\keywords{datasets, neural networks, gaze detection, text tagging}

%
% This command processes the author and affiliation and title information and builds
% the first part of the formatted document.
\maketitle

\section{Introduction}
%% 
%% Introduction
%%
%% Key ideas: 
%%
%% 1. Augmenting feature models with application domain modeling
%% 2. Impact this has on the toolsuite for featureIDE and composition engines
%%
%% 3. Expose programming API instead of providing a standard black-box
%%    capability, which might otherwise restrict users' abilities.

Software product lines (SPLs) refer to software engineering methods,
tools and techniques for creating a collection of similar software
systems, called product line members, from a shared set of software
assets using a common means of production. A number of characteristics
separate SPLs from routine large, complex systems and thus there is a
need for specialized tools and techniques to design, implement and
maintain SPLs.

Fundamental to many SPL approaches is the concept of a \textit{Feature
Model}, first introduced by the Feature-Oriented Domain Analysis (FODA)
technique~\cite{Kang1990} and refined over the years by numerous
researchers. A feature model is composed of user-visible features from
which a wide variety of configurations can be defined through the
selection of individual features. Many tool chains have been developed
to generate product line members ``on demand'' from a feature model
configuration. Developers need to construct the SPL code artifacts in a
specific way that enables the tools to work; this may include, for
example, low-level compiler directives embedded in source code, or
specialized languages (such as JAK~\cite{Batory2004FeatureorientedPA} or
DeltaJ~\cite{Schaefer:2010:DPS:1885639.1885647}) that developers use to
encode the artifacts.

%% ,natwidth=1154,natheight=576
%%\begin{figure}
%%  \includegraphics[scale=0.25]{gpl2.png}
%%  \caption{Graph Product Line Feature Model}
%%  \label{fig:feature-model}
%%\end{figure}

There remains a potential issue, however, when the variability inherent
in the product line relates to the application domain model. The common
result is that one must integrate these domain-model concepts into the
feature model so they can become part of the configuration process. The
feature model from Figure~\ref{fig:feature-model} contains features
relating to behavior (i.e., \textbf{Prim} contains Prim's implementation
of the Minimum Spanning Tree problem) and some relate to structure
(i.e., \textbf{Weighted} captures the notion that edges in a graph may
have weights), while a third set (i.e., \textbf{KruskalPrepGnR}) are
features that mediate between the behavior and structure. As one might
expect, since a Feature Model is global in scope (and uniformly
decomposed into features), the dominant decomposition can fail to
account for the interrelationships of the features.

In past work, we have explored product lines within a number of
application domains~\cite{Heineman:2015:TMO:2791060.2791076,PEPM18} as
we investigate the role that domain modeling should play within an SPLC.
In particular, given the extensive developments in model-driven
engineering, we looked into ways to integrate a separate domain model
into the tool chain used for generating product line instances. We
demonstrate how to incorporate domain modeling within the Graph Product
Line, a canonical SPLC used extensively within the research community.
Our focus is on using code generation to automatically generate the
boilerplate code that would otherwise have to be customized based on
structural considerations determined by the domain model. Our aim is to
make it easier to design the code artifacts for individual features that
will be able to compose together regardless of the structure of the
domain model. In a way, we are inspired by the underlying philosophy of
the Demeter project~\cite{karl:demeter} which aims to separate objects
from their operations, to allow each to evolve independently without
significant impact on each other.

%We make the following observations regarding product lines: % %--
%--Should be able to add to and remove features from a product line
 %after it has been defined  %--A product line definition is still
 %reflected in software artifacts, which means it should support common
 %refactoring functionality % %--Should be feature-rich, which means one
 %can potentially envision a significant number of PL members (e.g., not
 %just a handful) % %Without one of these characteristics, it could only
 %be considered a closed system.


% A SPL is a family of related programs. When the units of program
% construction are feature—increments in program functionality or
%development—every program in an SPL is identified by a unique and legal
%combination of features, and vice versa. Family member refers to
%individual product. A variation point represents a decision leading to
%different variants of an asset. A variation point consists of a set of
%possible instantiations (legal variations of the variation point). A
%variation point usually specifies the binding times, that is the
%time/times at which a decision about the instantiation has to be taken.

%The variant derivation is the action in which assets are combined from
%the set of available assets and contained variation points are
%bound/instantiated. If there are variation points with multiple binding
%times, the derivation will happen stepwise at each binding time. The
%result of such a derivation is a set of derived assets. The derivation
%can be executed technically in many ways. The simplest way is to copy
%assets and modify (parts of) them (e.g. the configuration parameters)
%manually. The result of such a derivation is often called a
%configuration.



%% THESE NEED TO BE PLACED IN CONTEXT. Here is a reference to
%K\"{a}stner's paper~\cite{Kastner:2012}. % %Here is a reference to
%~\cite{leich2005tool}. Here is a reference to %~\cite{proksch2014tool}.
%Here is a reference to ~\cite{7203038}. Here is %a reference to
%~\cite{Sayre:2005:UMA:1082983.1083277}. Here is a %reference to
%~\cite{Setyautami:2016:UPD:2934466.2934479}. Here is a %reference to
%~\cite{Sousa:2016:EFM:2934466.2934475}. % %Here is a reference to
%~\cite{Arcaini:2017:ARV:3106195.3106206}. % %Here is a reference to
%~\cite{Vasilevskiy:2016:TRP:2934466.2934484}. % %Here is a reference to
%~\cite{Kuhn:2015:CPC:2791060.2791092}. % %Here is a reference to
%~\cite{Schaefer:2010:DPS:1885639.1885647}. % %Here is a reference to
%~\cite{Apel:2008:AFF:1428476.1428480}.


\subsection{Approaches to Product Line Development}

Software product lines create unique engineering challenges for a number
of reasons. First, it should be possible to generate any number of
instances of the product line, and by this we mean the code artifacts
for the instance application. These are then compiled (or interpreted)
to realize the execution of the application. Second, there are multiple
development efforts; one can work on code that will effectively be used
by all product line instances, but at other times, one is focused on
writing code for just a single instance from the PL.

A major goal of featured-oriented PL is to derive a product
automatically based on user’s selection. There are three philosophical
approaches widely used in practice -- \textit{annotation-based} ,
\textit{composition-based} or \textit{component-based} -- which differ
in the way they represent variability in the code base and the tool
chains used to generate the product line members.

\subsubsection{Annotation-based Approach}

An annotation-based SPLC consists of a single body of code artifacts
which fully contains all code resources used by all members of the
product line. Using different tools or language-specific capabilities, a
compiler (or pre-processor) extracts subsets of the code to be used for
a PL instance. One of the most common approaches is to use compiler
directives embedded within the code as a means for isolating code unique
to a subset of product line instances. Then each product line instance
can be generated by compiling the same code base with different compiler
flags, resulting in different executable instances. Due to the nature of
this approach, often one cannot review the source code for individual
instances~\cite{Apel:2013:FSP:2541773}. Often the compiler directives
manipulate the core data structures or class definitions of the
software, for example, adding or removing a class field
definition~\cite{Liebig:2010:AVF:1806799.1806819}.

Annotation-based approaches are widely used in practice because they are
easy to use and are already natively supported by many programming
environments. While these approaches are easily adopted, the limited
(and typically non-extensible) tool support can make it more complicated
to maintain an SPLC in the face of constant evolution and added (or
removed) features. A product-line-aware type checking
system has also been developed to statically and efficiently detects
type errors in annotation-based product-line.
implementations~\cite{Kastner:2012,Apel:2013:FSP:2541773}.

\subsubsection{Composition-based Approach}

Another common approach is to design a \textit{feature model tree} to capture the
externally visible features that differentiate one product line instance
from another. Each feature can include any number of associated code
artifacts. A product line instance is configured by selecting
for inclusion features from the feature tree, potentially restricted by
constraints. A composition engine processes the code assets associated
with the selected features to create the final source code for the
product line instance.

Composition-based approaches locate code belonging to a feature or
feature combination in a dedicated file, container, or module. A classic
example is a framework that can be extended with plug-ins, ideally one
plug-in per feature. The key challenge of composition-based approaches
is to keep the mapping between features and composition units simple and
tractable.

Another way to view the difference between annotation and composition is
that annotation separate concerns virtually and composition separate
concerns physically,and code is removed on demand with annotation while
composition units are added on demand.
~\cite{Thum:2014:FEF:2537169.2537315,Apel:2013:FSP:2541773}.

\subsubsection{Component-based Approach}

A software component is a unit of composition with contractually
specified interfaces and explicit context dependencies only. A software
component can be deployed independently and is subject to composition by
third parties. The key idea of a component is to form a modular,
reusable unit.

Developing product line by constructing and composing reusable was a
common strategy. With domain analysis, developers decided which
functionality should be reused across multiple products of the product
line designed components accordingly. To derive a product for a given
feature selection during application engineering, a developer selects
suitable components and then manually writes glue code to connect
components for every product individually.



\subsubsection{Product Line Techniques In Industry}

Annotation-based approach are not so widely used as composition-based
approach because drawbacks mentioned above. Here are some existing
annotation-based approaches:

CIDE~\cite{CIDE:Eclipse} is an Eclipse plug-in that
replaces the Java editor in SPL projects. Developers start with a
standard Java legacy application, then they select code fragment and
associate them with features from the context menu. The marked code is
then permanently highlighted in the editor using a background color
associated with the feature. Researchers have shown how to type check
software product lines to identify and eliminate common errors, such as
missing class fields or methods.

Boost is a set of libraries for the C++ programming language that provide support for
tasks and structures.


To work with FeatureIDE, the primary challenge is to design a feature
tree model to represent the desired product line application domain.
Because features are cross-cutting with regards to the artifacts in the
programming language, the various composer engines supported by
FeatureIDE accomplish the same goal in a variety of ways.

AHEAD has feature modules for each concrete feature, and the
corresponding composition tool places generated source code directly
into the Eclipse source folder. AHEAD brings separate tools together and
selects different tools for different kinds of files during feature
composition,establishing a clear interface to the build system.
Composing Jak files will invoke a Jak-composition, whereas composing XML
files invokes an XML-composition
tool.~\cite{Batory2004FeatureorientedPA}.

FeatureHouse tool suite has been developed that allows programmers to
enhance given languages rapidly with support for feature-oriented
programming. It is a framework for software composition supported by a
corresponding tool chain. It provides facilities for feature composition
based on a language-independent model and tool chain for software
artifact, and a plug-in mechanism for the integration of new artifact
languages.~\cite{Apel:2009:FLA:1555001.1555038}

Deltaj is a Java-like language which allows to organize classes in
modules. A program consists of a base module and a set of delta module
in a stepwise manner. Much like a feature module, a delta module can add
new classes and members as well as extend existing methods by
overriding. In contrast to feature modules, delta modules can also
delete existing classes and individual
members.~\cite{Schaefer:2010:DPS:1885639.1885647}.

In LaunchPad, each feature can contain any number of combinators,
designed using a DSL we had developed to simplify the writing of
combinators for an earlier CLS tool. A configuration is a subset of
features from which a repository is constructed. Each feature can
optionally store target definitions, which are aggregated together and
then used as the basis for the inhabitation requests, i.e.As each
request is satisfied, the synthesized code from the resulting type
expression M is stored in the designated source folder.
~\cite{Heineman:2015:TMO:2791060.2791076}.

\subsubsection{Evaluation of related work}

The key challenge of composition-based approaches is to keep the mapping
between features and composition units simple and tractable.
Preprocessor-based and parameter-based implementations are often
criticized for their potential complexity, lack of modularity, and
reduced readability. And they all have some problems which is the
difficulty to refactoring.~\cite{Kim:2017:RJS:3106195.3106201}.

Although component-based implementations are common in product line
practice, they lack the automation potential of feature orientation that
we aim at. Deciding when to build a reusable component and what to
include in that component is a difficult design decision, one need to
have good understanding of whole scope to do that. There is no domain in
this approach, glue code is a necessary, it's no more than assembling
components. ~\cite{Apel:2013:FSP:2541773}

In practice most PLs appear “from the ground up” where developers take
advantage of language-specific capabilities to annotate different code
regions as being enabled (or disabled) based on compiler directives.
Starting from an annotation-based code repository many composition-based
approaches simply “snipped” or refactored code fragments to recreate
countless tiny “features” that could be selected.

Manual composition is a configuration process. A designer selects
individual features from a feature model and relies on constraints to
ensure the resulting product line member is valid. Manual Composition is
limited to a potential total of 2N configurations where N represents the
number of available features in the model. There is no domain modeling,
What commonly occurs is the designer must make sure that changes to any
of the units will not invalidate those product line members that
incorporate that feature.

Another problem is that the features are fixed and unchanging. If we
need to make some modifications to current instances, we may need to
trace all the way back and change the code in many classes because it’s
inheritance structure. If we want to add features which is slightly
different from existing ones, we may need to start from very beginning.


\section{A New Approach is Needed}

There are features to represent the structure of a variations, and there is a feature for each variation.
Our goal was to support the easy construction of new variations by reusing existing features where possible
and adding new features as needed to support the functionality expected of new variations.


\subsection{Essential Characteristics}

\subsubsection{CLS generic composition}

With dominated approach for using feature in PL, n features in the feature tree may generate 2n configurations
 which will become product line instances. But if we use CLS as the algorithm for composition, the fundamental
 units will be combinators instead of features. The CLS starts with a repository of combinators to which a user
 issues a query which attempts to find a type in the repository using inferencing.

Combinators can be dynamic and added at composition time, something which is simply not possible in traditional
feature trees used by feature-oriented product lines.

To better explain these dynamic combinators, consider having a feature model with a feature that provides
variability and there are a number of fixed sub-features that are tailored for each valid variation.
For example, “Number of external hard disks” might have sub-features “One-Hard-Disk”, “Two-hard-disks” and
so on. Individual members of the PL are configured, accordingly, to select the desired number of external hard
disks. In contrast, using CLS a single combinator class NumberOfExternalDisks is parameterized with an integer,
and one can instantiate a combinator (NumberOfExternalDisks(3)) and add to the repository as needed based on
the modeling needs of the member.

Without making 2n configurations, using CLS will significantly simplify code system in PL, optimize code structure
make it more readable and reasonable.

\subsubsection{Application domain modeling}

This appears missing in nearly every approach we see. This happens because Feature Trees do not require any
other modeling besides the tree itself, and annotation-based approaches rely solely on the codebase itself.

Because there is no domain modeling, the various Feature-IDE approaches all appear to have configurations
which become ineffective domain modeling. For example, in some FeatureIDE models,
there is a feature with sub-features that appear to be nothing more than instantiations of different
configurations, which makes the width of feature tree in AHEAD huge.

\subsubsection{Compositional manipulation}

Feature-IDE relies on externally provided composition engines to process code fragments. The challenge
is that FeatureIDE can make no semantic guarantees about the resulting code. Also there is no theoretical
foundation for the composition, which rather simply is assembly. During assembly, units are wired together
without making any changes to the units themselves.

\subsubsection{Language agnostic}


Without being language limited as normal ways, our approach is more laguage agnostic. Choice of language have
been more diversified, which could benefit more engineers with different backgrounds. We don't have to take
advantage of single language to build our code base. For example,we have to use .jak files in AHEAD. If
you are not familiar with the language, you can't use the approach.

\subsubsection{Code sharing between assets}


Like we mentioned above, dynamic combinators can be constructed to add methods into classes. Assets can share
some basic code, with different methods included.

\section{Graph Product Line}
%% Here is where to describe GPL


The Graph Product Line (GPL) is a well-known case study within the software product line community
~\cite{Lopez-Herrejon:2001:SPE:645418.652082}.Graph is a fundamental topic in computer science and choosing it minimises the requirement
of becoming a domain-expert which is a prerequisite in the product line development process. Still it is complex
enough to expose the underlying concepts of product lines and their implementation. This product line
 supports variations in a library of graph data strutures and algorithms. A possible feature diagram of the graph
 library is shown in Figure?.Like any other product line, members of the GPL share some common features and differ in
 certain variant features. The root is labeled with Gpl to represent a graph product. It has a mandatory child feature
 GraphType, because each graph library has to implement an type, which is either Directed or Undirected. Furthermore,
 three other child features of the root are optional: Search, Weighted and Algorithm. Search strategies may be either
  breadth-first search (BFS) or depth-first search (DFS). Algorithm offers a selection of graph algorithms as child
  features. Since it's optional, either zero, one, ore more algorithms may be presented in a graph product. In our
  example, the algorithm for minimal spanning trees MST has two alternative implementations, Prim and Kruskal. Some
  non-local conditions are modeled as explicit Boolean constraints-- for example, minimal spanning tree make only
  sense for weighted graphs, and shortest paths can be computed directed graphs only.

  Some of the common and variant features that the GPL members share are identified as
  follows:

  • Common Features: The common features that the GPL applications share are Vertex and Edge.
  Which means a GPL application must have these features.

  • Variant Features: Members across the GPL share some variant features:

  – Graph Type: A graph is either directed or undirected.
  – Weight: Edges in a graph can be weighted with non-negative numbers or unweighted.
  – Search: A graph application can implement at most one searching algorithm, Breadth First
  search (BFS), Depth First Search (DFS) or none.
  – Algorithms: A graph application implements one or more of the following algorithms: Number
  (Vertex Numbering), Connected (Connected Components), Strongly (Strongly Connected
  Components), Cycle (Cycle Checking), Shortest (Single-Source Shortest Path) and MST(Minimum
  Spanning Tree).Details of these algorithms can be found in any algorithm book and more information
  on GPL can be found at: http://www.cs.utexas.edu/users/dsb/GPL/graph.htm.
  AND http://stg-tud.github.io/sedc/Lecture/ws16-17/6-SPL.pdf.

Here are papers using GPL ~\cite{Thum:2011:AFF:2061045.2062153}.
~\cite{ZhangGraphPL}.
~\cite{LopezHerrejon2014TowardsAB}.
~\cite{Bagheri:2010:CSP:1885639.1885642}.
~\cite{Lopez-Herrejon:2012:TFI:2110147.2110158}.
~\cite{{Johansen2010ExploringTS}.
~\cite{doi:10.1002/smr.534}.
~\cite{10.1007/978-3-642-31095-9_40}.



\section{Design of CLS-based GPL}
The design of the CLS-based GPL goes here...

\section{Evaluation}
Evalution of our approach goes here...

\section{Conclusion}
In our approach, all product line family modeling is accomplished in the application domain model space and the
 combinators are written to simply “materialize” the actual solution-domain code that reflects the application
 domain model. This work will also enable true independence from underlying technologies and external
 dependencies, and in particular, address the common problem of upgrading code that depends upon a framework
 that has evolved.




\section{Acknowledgments}

Identification of funding sources and other support, and thanks to individuals and groups that assisted in the research and the preparation of the work should be included in an acknowledgment section, which is placed just before the reference section in your document. 

This section has a special environment:
\begin{verbatim}
  \begin{acks}
  ...
  \end{acks}
\end{verbatim}
so that the information contained therein can be more easily collected during the article metadata extraction phase, and to ensure consistency in the spelling of the section heading. 

Authors should not prepare this section as a numbered or unnumbered {\verb|\section|}; please use the ``{\verb|acks|}'' environment.

\section{Appendices}

If your work needs an appendix, add it before the ``\verb|\end{document}|'' command at the conclusion of your source document. 

Start the appendix with the ``\verb|appendix|'' command:
\begin{verbatim}
  \appendix
\end{verbatim}
and note that in the appendix, sections are lettered, not numbered. This document has two appendices, demonstrating the section and subsection identification method.

\section{SIGCHI Extended Abstracts}

The ``\verb|sigchi-a|'' template style (available only in \LaTeX\ and not in Word) produces a landscape-orientation formatted article, with a wide left margin. Three environments are available for use with the ``\verb|sigchi-a|'' template style, and produce formatted output in the margin:
\begin{itemize}
\item {\verb|sidebar|}:  Place formatted text in the margin.
\item {\verb|marginfigure|}: Place a figure in the margin.
\item {\verb|margintable|}: Place a table in the margin.
\end{itemize}

%
% The acknowledgments section is defined using the "acks" environment (and NOT an unnumbered section). This ensures
% the proper identification of the section in the article metadata, and the consistent spelling of the heading.
\begin{acks}
To Robert, for the bagels and explaining CMYK and color spaces.
\end{acks}

%
% The next two lines define the bibliography style to be used, and the bibliography file.
\bibliographystyle{ACM-Reference-Format}
\bibliography{splc19}

% 
% If your work has an appendix, this is the place to put it.
\appendix

\section{Research Methods}

\subsection{Part One}

Lorem ipsum dolor sit amet, consectetur adipiscing elit. Morbi malesuada, quam in pulvinar varius, metus nunc fermentum urna, id sollicitudin purus odio sit amet enim. Aliquam ullamcorper eu ipsum vel mollis. Curabitur quis dictum nisl. Phasellus vel semper risus, et lacinia dolor. Integer ultricies commodo sem nec semper. 

\subsection{Part Two}

Etiam commodo feugiat nisl pulvinar pellentesque. Etiam auctor sodales ligula, non varius nibh pulvinar semper. Suspendisse nec lectus non ipsum convallis congue hendrerit vitae sapien. Donec at laoreet eros. Vivamus non purus placerat, scelerisque diam eu, cursus ante. Etiam aliquam tortor auctor efficitur mattis. 

\section{Online Resources}

Nam id fermentum dui. Suspendisse sagittis tortor a nulla mollis, in pulvinar ex pretium. Sed interdum orci quis metus euismod, et sagittis enim maximus. Vestibulum gravida massa ut felis suscipit congue. Quisque mattis elit a risus ultrices commodo venenatis eget dui. Etiam sagittis eleifend elementum. 

Nam interdum magna at lectus dignissim, ac dignissim lorem rhoncus. Maecenas eu arcu ac neque placerat aliquam. Nunc pulvinar massa et mattis lacinia.

\end{document}

%
% The first command in your LaTeX source must be the \documentclass command.
\documentclass[sigconf]{acmart}

%
% defining the \BibTeX command - from Oren Patashnik's original BibTeX documentation.
\def\BibTeX{{\rm B\kern-.05em{\sc i\kern-.025em b}\kern-.08emT\kern-.1667em\lower.7ex\hbox{E}\kern-.125emX}}
    
% Rights management information. 
% This information is sent to you when you complete the rights form.
% These commands have SAMPLE values in them; it is your responsibility as an author to replace
% the commands and values with those provided to you when you complete the rights form.
%
% These commands are for a PROCEEDINGS abstract or paper.
\copyrightyear{2018}
\acmYear{2018}
\setcopyright{acmlicensed}

\acmConference[SPLC'19]{23rd International Systems and Software Product Line Conference}{9--13 September, 2019}{Paris, France}
\acmBooktitle{Woodstock '18: ACM Symposium on Neural Gaze Detection, June 03--05, 2018, Woodstock, NY}
\acmPrice{15.00}
\acmDOI{10.1145/1122445.1122456}
\acmISBN{978-1-4503-9999-9/18/06}

%
% These commands are for a JOURNAL article.
%\setcopyright{acmcopyright}
%\acmJournal{TOG}
%\acmYear{2018}\acmVolume{37}\acmNumber{4}\acmArticle{111}\acmMonth{8}
%\acmDOI{10.1145/1122445.1122456}

%
% Submission ID. 
% Use this when submitting an article to a sponsored event. You'll receive a unique submission ID from the organizers
% of the event, and this ID should be used as the parameter to this command.
%\acmSubmissionID{123-A56-BU3}

%
% The majority of ACM publications use numbered citations and references. If you are preparing content for an event
% sponsored by ACM SIGGRAPH, you must use the "author year" style of citations and references. Uncommenting
% the next command will enable that style.
%\citestyle{acmauthoryear}

%
% end of the preamble, start of the body of the document source.
\begin{document}

%
% The "title" command has an optional parameter, allowing the author to define a "short title" to be used in page headers.
\title{Towards Automated Composition of a Product Line}

%
% The "author" command and its associated commands are used to define the authors and their affiliations.
% Of note is the shared affiliation of the first two authors, and the "authornote" and "authornotemark" commands
% used to denote shared contribution to the research.
\author{Shengmei Liu}
\email{sliu7@wpi.edu}
\affiliation{%
  \institution{WPI}
  \streetaddress{100 Institute Road}
  \city{Worcester}
  \state{Massachusetts}
  \postcode{01609}
}

\author{George T. Heineman}
\email{heineman@wpi.edu}
\affiliation{%
  \institution{WPI}
  \streetaddress{100 Institute Road}
  \city{Worcester}
  \state{Massachusetts}
  \postcode{01609}
}

%
% By default, the full list of authors will be used in the page headers. Often, this list is too long, and will overlap
% other information printed in the page headers. This command allows the author to define a more concise list
% of authors' names for this purpose.
% \renewcommand{\shortauthors}{Trovato and Tobin, et al.}

%
% The abstract is a short summary of the work to be presented in the article.
\begin{abstract}
Object-oriented (OO) frameworks represent a significant achievement in extensible design,
but there are many well-documented challenges when third-party programmers attempt to use and
refactor them. In earlier work, we described how to migrate existing OO framework- based software
into a software product line structure using combinatory logic synthesis (CLS) integrated into FeatureIDE,
an Eclipse-based IDE that supports feature-oriented software development. While initially successful at
synthesizing a few instances of a product line, the approach does not scale to support larger product
lines because it does not adequately capture the commonality and inherent variability in the application domain.
In this paper, we analyze these problems, and introduce our new approach to construct product line which
overcomes many drawbacks of old approach. We describe how application domain modeling helps automated
composition, and how our redesigned CLS-engine improves scalability. Results are illustrated by scaling
the well-known graph product line with our approach, while reducing the average amount of instance-specific
code significantly, generating more readable code and providing more convenience for refactoring.
\end{abstract}

%
% The code below is generated by the tool at http://dl.acm.org/ccs.cfm.
% Please copy and paste the code instead of the example below.
%

%
% Keywords. The author(s) should pick words that accurately describe the work being
% presented. Separate the keywords with commas.
\keywords{datasets, neural networks, gaze detection, text tagging}

%
% This command processes the author and affiliation and title information and builds
% the first part of the formatted document.
\maketitle

\section{Introduction}
\subsection{Definitions and Terms}

Software product lines (SPLs) refer to software engineering methods, tools and
techniques for creating a collection of similar software systems from a shared set of
software assets using a common means of production.

A SPL is a family of related programs. When the units of program construction are features—increments in
program functionality or development—every program in an SPL is identified by a unique and legal
combination of features, and vice versa. Family member refers to individual product. A variation point
represents a decision leading to different variants of an asset. A variation point consists of a set of
possible instantiations (legal variations of the variation point). A variation point usually specifies the
binding times, that is the time/times at which a decision about the instantiation has to be taken.

The variant derivation is the action in which assets are combined from the set of available assets and
contained variation points are bound/instantiated. If there are variation points with multiple binding times,
the derivation will happen stepwise at each binding time. The result of such a derivation is a set of derived
assets. The derivation can be executed technically in many ways. The simplest way is to copy assets and modify
(parts of) them (e.g. the configuration parameters) manually. The result of such a derivation is often called a
configuration.

We make the following observations regarding product lines:

--It should be possible to generate an instance of the product line using configuration

--Should be able to add to and remove features from a product line after it has been defined

--A product line definition is still reflected in software artifacts, which means it should support common refactoring functionality

--Should be feature-rich, which means one can potentially envision a significant number of PL members (e.g., not just a handful)

Without one of these characteristics, it could only be considered a closed system.

Here is a reference to K\"{a}stner's paper~\cite{Kastner:2012}.

Here is a reference to ~\cite{leich2005tool}.
Here is a reference to ~\cite{Thum:2014:FEF:2537169.2537315}.
Here is a reference to ~\cite{proksch2014tool}.
Here is a reference to ~\cite{7203038}.
Here is a reference to ~\cite{Sayre:2005:UMA:1082983.1083277}.
Be sure to place where it is discussed~\cite{CIDE:Eclipse}.

\subsection{Approaches to Product Line Development}

Software product lines create unique engineering challenges for a number of reasons. First, it should be
possible to generate any number of  instances of the product line, and by this we mean the code artifacts
for the instance application. These are then compiled (or interpreted) to realize the execution of the
application. Second, there are multiple development efforts; one can work on code that will effectively be
used by all product line instances, but at other times, one is focused on writing code for just a single
instance from the PL.

A major goal of featured-oriented PL is to derive a product automatically based on user’s selection. There are
two approach widely used in practice,  annotation-based approach or a composition-based approach, which differ in
the way they represent variability in the code base and the way they generate products.


\subsubsection{Annotation-based Approach}

There is a single body of code artifacts which fully contains all code resources used by all members of the product line.
Using different tools or language-specific capabilities, a compiler (or processor) will extract subsets of the code to be used
 for a PL instance. One of the most common approaches is to use compiler directives embedded within the code as a means
 for isolating code unique to a subset of product line instances. Then each product line instance can be generated
 by compiling the same code base with different compiler flags, resulting in different executable instances.
 Due to the nature of this approach, often one cannot review the source code for individual instances.

In annotation-based approaches, the code of all features is merged in a single code base, and annotation mark which code
 belongs to which feature. In some sense, an annotation is a function that maps a program element to the feature or
 feature combination it belongs to.

Annotation-based approaches are widely used in practice because they are easy to use and already natively supported
by many programming environments. It keeps good readability and low complexity, however, relatively simple tool support
can address scattered code or error.




\subsubsection{Composition-based Approach}

Another approach is to design a feature tree which is used to capture all the externally visible features that
can be used to differentiated one product line instance from another. Then code assets are internally associated
 with each of these visible features. Finally product line instances are configured by selecting for inclusion
  features from the feature tree, potentially restricted by constraints. A composition engine processes the
  code assets associated with the selected features to create the final source code for the product line instance.

Composition-based approaches locate code belonging to a feature or feature combination in a dedicated file,
container, or module. A classic example is a framework that can be extended with plug-ins, ideally one plug-in
per feature. The key challenge of composition-based approaches is to keep the mapping between features and
composition units simple and tractable.

Another way to view the difference between annotation and composition is that annotation separate concerns
virtually and composition separate concerns physically,and code is removed on demand with annotation while
composition units are added on demand.


\subsubsection{Product Line Techniques In Industry}

Annotation-based approach are not so widely used as composition-based approach because drawbacks mentioned above.
Here are some existing annotation-based approaches:

code coloring (FeatureCIDE): CIDE is an Eclipse plug-in that replaces the Java editor in SPL projects. Developers
start with a standard Java legacy application, then they select code fragment and associate them with features
 from the context menu. The marked code is then permanently highlighted in the editor using a background color
 associated with the feature

Type checking approach,  a product-line-aware type system that statically and efficiently detects type errors in
annotation-based product-line implementations.


To work with FeatureIDE, the primary challenge is to design a feature tree model to represent the desired product
line application domain. Because features are cross-cutting with regards to the artifacts in the programming
language, the various composer engines supported by FeatureIDE accomplish the same goal in a variety of ways.

AHEAD has feature modules for each concrete feature, and the corresponding composition tool places generated
source code directly into the Eclipse source folder. AHEAD brings separate tools together and selects different
tools for different kinds of files during feature composition,establishing a clear interface to the build
system. Composing Jak files will invoke a Jak-composition, whereas composing XML files invokes an
XML-composition tool.

FeatureHouse tool suite has been developed that allows programmers to enhance given languages rapidly with
support for feature-oriented programming. It is a framework for software composition supported by a
corresponding tool chain. It provides facilities for feature  composition based on a language-independent
model and tool chain for software artifact, and a plug-in mechanism for the integration of new artifact
languages.

Deltaj is a Java-like language which allows to organize classes in modules. A program consists of a base
 module and a set of delta module in a stepwise manner. Much like a feature module, a delta module can add
 new classes and members as well as extend existing methods by overriding. In contrast to feature modules,
 delta modules can also delete existing classes and individual members.

In LaunchPad, each feature can contain any number of combinators, designed using a DSL we had developed to
simplify the writing of combinators for an earlier CLS tool. A configuration is a subset of features from
which a repository is constructed. Each feature can optionally store target definitions, which are aggregated
 together and then used as the basis for the inhabitation requests, i.e.As each request is satisfied, the
 synthesized code from the resulting type expression M is stored in the designated source folder.


\subsubsection{Evaluation of related work}

The key challenge of composition-based approaches is to keep the mapping between features and composition units
simple and tractable. Preprocessor-based and parameter-based implementations are often criticized for their
potential complexity, lack of modularity, and reduced readability. And they all have some problems which is the
hard to refactoring.

In practice most PLs appear “from the ground up” where developers take advantage of language-specific capabilities
 to annotate different code regions as being enabled (or disabled) based on compiler directives. Starting from an
 annotation-based code repository many composition-based approaches simply “snipped” or refactored code fragments
 to recreate countless tiny “features” that could be selected.

Manual composition is a configuration process. A designer selects individual features from a feature model and
relies on constraints to ensure the resulting product line member is valid. Manual Composition is limited to a
potential total of 2N configurations where N represents the number of available features in the model. There is no
 domain modeling, What commonly occurs is the designer must make sure that changes to any of the units will not invalidate those product
 line members that incorporate that feature.

 Another problem is that the features are fixed and unchanging. If we need to make some modifications to current
 instances, we may need to trace all the way back and change the code in many classes because it’s inheritance
 structure. If we want to add features which is slightly different from existing ones, we may need to start from
 very beginning.

\section{A New Approach is Needed}

There are features to represent the structure of a variations, and there is a feature for each variation.
Our goal was to support the easy construction of new variations by reusing existing features where possible
and adding new features as needed to support the functionality expected of new variations.


\subsection{Essential Characteristics}

\subsubsection{CLS generic composition}

With dominated approach for using feature in PL, n features in the feature tree may generate 2n configurations
 which will become product line instances. But if we use CLS as the algorithm for composition, the fundamental
 units will be combinators instead of features. The CLS starts with a repository of combinators to which a user
 issues a query which attempts to find a type in the repository using inferencing.

Combinators can be dynamic and added at composition time, something which is simply not possible in traditional
feature trees used by feature-oriented product lines.

To better explain these dynamic combinators, consider having a feature model with a feature that provides
variability and there are a number of fixed sub-features that are tailored for each valid variation.
For example, “Number of external hard disks” might have sub-features “One-Hard-Disk”, “Two-hard-disks” and
so on. Individual members of the PL are configured, accordingly, to select the desired number of external hard
disks. In contrast, using CLS a single combinator class NumberOfExternalDisks is parameterized with an integer,
and one can instantiate a combinator (NumberOfExternalDisks(3)) and add to the repository as needed based on
the modeling needs of the member.

Without making 2n configurations, using CLS will significantly simplify code system in PL, optimize code structure
make it more readable and reasonable.

\subsubsection{Application domain modeling}

This appears missing in nearly every approach we see. This happens because Feature Trees do not require any
other modeling besides the tree itself, and annotation-based approaches rely solely on the codebase itself.

Because there is no domain modeling, the various Feature-IDE approaches all appear to have configurations
which become ineffective domain modeling. For example, in some FeatureIDE models,
there is a feature with sub-features that appear to be nothing more than instantiations of different
configurations, which makes the width of feature tree in AHEAD huge.

\subsubsection{Compositional manipulation}

Feature-IDE relies on externally provided composition engines to process code fragments. The challenge
is that FeatureIDE can make no semantic guarantees about the resulting code. Also there is no theoretical
foundation for the composition, which rather simply is assembly. During assembly, units are wired together
without making any changes to the units themselves.

\subsubsection{Language agustic}


Without being language limited as normal ways, our approach is more laguage agustic. Choice of language have
been more diversified, which could benefit more engineers with different backgrounds. We don't have to take
advantage of single language to build our code base. For example,we have to use .jak files in AHEAD. If
you are not familiar with the language, you can't use the approach.

\subsubsection{Code sharing between assets}


Like we mentioned above, dynamic combinators can be constructed to add methods into classes. Assets can share
some basic code, with different methods included.

\subsection{Well-known graph product line}




\section{Template Overview}
As noted in the introduction, the ``\verb|acmart|'' document class can be used to prepare many different kinds of documentation --- a double-blind initial submission of a full-length technical paper, a two-page SIGGRAPH Emerging Technologies abstract, a ``camera-ready'' journal article, a SIGCHI Extended Abstract, and more --- all by selecting the appropriate {\it template style} and {\it template parameters}.

This document will explain the major features of the document class. For further information, the {\it \LaTeX\ User's Guide} is available from \url{https://www.acm.org/publications/proceedings-template}.

\subsection{Template Styles}

The primary parameter given to the ``\verb|acmart|'' document class is the {\it template style} which corresponds to the kind of publication or SIG publishing the work. This parameter is enclosed in square brackets and is a part of the {\verb|documentclass|} command:
\begin{verbatim}
  \documentclass[STYLE]{acmart}
\end{verbatim}

Journals use one of three template styles. All but three ACM journals use the {\verb|acmsmall|} template style:
\begin{itemize}
\item {\verb|acmsmall|}: The default journal template style.
\item {\verb|acmlarge|}: Used by JOCCH and TAP.
\item {\verb|acmtog|}: Used by TOG.
\end{itemize}

The majority of conference proceedings documentation will use the {\verb|acmconf|} template style.
\begin{itemize}
\item {\verb|acmconf|}: The default proceedings template style.
\item{\verb|sigchi|}: Used for SIGCHI conference articles.
\item{\verb|sigchi-a|}: Used for SIGCHI ``Extended Abstract'' articles.
\item{\verb|sigplan|}: Used for SIGPLAN conference articles.
\end{itemize}

\subsection{Template Parameters}

In addition to specifying the {\it template style} to be used in formatting your work, there are a number of {\it template parameters} which modify some part of the applied template style. A complete list of these parameters can be found in the {\it \LaTeX\ User's Guide.}

Frequently-used parameters, or combinations of parameters, include:
\begin{itemize}
\item {\verb|anonymous,review|}: Suitable for a ``double-blind'' conference submission. Anonymizes the work and includes line numbers. Use with the \verb|\acmSubmissionID| command to print the submission's unique ID on each page of the work.
\item{\verb|authorversion|}: Produces a version of the work suitable for posting by the author.
\item{\verb|screen|}: Produces colored hyperlinks.
\end{itemize}

This document uses the following string as the first command in the source file: \verb|\documentclass[sigconf,screen]{acmart}|.

\section{Modifications}

Modifying the template --- including but not limited to: adjusting margins, typeface sizes, line spacing, paragraph and list definitions, and the use of the \verb|\vspace| command to manually adjust the vertical spacing between elements of your work --- is not allowed.

{\bf Your document will be returned to you for revision if modifications are discovered.}

\section{Typefaces}

The ``\verb|acmart|'' document class requires the use of the ``Libertine'' typeface family. Your \TeX\ installation should include this set of packages. Please do not substitute other typefaces. The ``\verb|lmodern|'' and ``\verb|ltimes|'' packages should not be used, as they will override the built-in typeface families.

\section{Title Information}

The title of your work should use capital letters appropriately - \url{https://capitalizemytitle.com/} has useful rules for capitalization. Use the {\verb|title|} command to define the title of your work. If your work has a subtitle, define it with the {\verb|subtitle|} command.
Do not insert line breaks in your title.

If your title is lengthy, you must define a short version to be used in the page headers, to prevent overlapping text. The \verb|title| command has a ``short title'' parameter:
\begin{verbatim}
  \title[short title]{full title}
\end{verbatim}

\section{Authors and Affiliations}

Each author must be defined separately for accurate metadata identification. Multiple authors may share one affiliation. Authors' names should not be abbreviated; use full first names wherever possible. Include authors' e-mail addresses whenever possible.

Grouping authors' names or e-mail addresses, or providing an ``e-mail alias,'' as shown below, is not acceptable:
\begin{verbatim}
  \author{Brooke Aster, David Mehldau}
  \email{dave,judy,steve@university.edu}
  \email{firstname.lastname@phillips.org}
\end{verbatim}

The \verb|authornote| and \verb|authornotemark| commands allow a note to apply to multiple authors --- for example, if the first two authors of an article contributed equally to the work. 

If your author list is lengthy, you must define a shortened version of the list of authors to be used in the page headers, to prevent overlapping text. The following command should be placed just after the last \verb|\author{}| definition:
\begin{verbatim}
  \renewcommand{\shortauthors}{McCartney, et al.}
\end{verbatim}
Omitting this command will force the use of a concatenated list of all of the authors' names, which may result in overlapping text in the page headers.

The article template's documentation, available at
\url{https://www.acm.org/publications/proceedings-template}, has a
complete explanation of these commands and tips for their effective use.

\section{Rights Information}

Authors of any work published by ACM will need to complete a rights form. Depending on the kind
of work, and the rights management choice made by the author, this may be copyright transfer,
permission, license, or an OA (open access) agreement.

Regardless of the rights management choice, the author will receive a copy of the completed rights form once it has been submitted. This form contains \LaTeX\ commands that must be copied into the source document. When the document source is compiled, these commands and their parameters add formatted text to several areas of the final document:
\begin{itemize}
\item the ``ACM Reference Format'' text on the first page.
\item the ``rights management'' text on the first page.
\item the conference information in the page header(s).
\end{itemize}

Rights information is unique to the work; if you are preparing several works for an event, make sure to use the correct set of commands with each of the works.

\section{CCS Concepts and User-Defined Keywords}

Two elements of the ``acmart'' document class provide powerful taxonomic tools for you to help readers find your work in an online search. 

The ACM Computing Classification System --- \url{https://www.acm.org/publications/class-2012} --- is a set of classifiers and concepts that describe the computing discipline. Authors can select entries from this classification system, via \url{https://dl.acm.org/ccs/ccs.cfm}, and generate the commands to be included in the \LaTeX\ source. 

User-defined keywords are a comma-separated list of words and phrases of the authors' choosing, providing a more flexible way of describing the research being presented.

CCS concepts and user-defined keywords are required for all short- and full-length articles, and optional for two-page abstracts. 

\section{Sectioning Commands}

Your work should use standard \LaTeX\ sectioning commands: \verb|section|, \verb|subsection|, \verb|subsubsection|, and \verb|paragraph|. They should be numbered; do not remove the numbering from the commands. 

Simulating a sectioning command by setting the first word or words of a paragraph in boldface or italicized text is {\bf not allowed.}

\section{Tables}

The ``\verb|acmart|'' document class includes the ``\verb|booktabs|'' package --- \url{https://ctan.org/pkg/booktabs} --- for preparing high-quality tables. 

Table captions are placed {\it above} the table.

Because tables cannot be split across pages, the best placement for them is typically the top of the page nearest their initial cite.  To ensure this proper ``floating'' placement of tables, use the environment \textbf{table} to enclose the table's contents and the table caption.  The contents of the table itself must go in the \textbf{tabular} environment, to be aligned properly in rows and columns, with the desired horizontal and vertical rules.  Again, detailed instructions on \textbf{tabular} material are found in the \textit{\LaTeX\ User's Guide}.

Immediately following this sentence is the point at which Table~\ref{tab:freq} is included in the input file; compare the placement of the table here with the table in the printed output of this document.

\begin{table}
  \caption{Frequency of Special Characters}
  \label{tab:freq}
  \begin{tabular}{ccl}
    \toprule
    Non-English or Math&Frequency&Comments\\
    \midrule
    \O & 1 in 1,000& For Swedish names\\
    $\pi$ & 1 in 5& Common in math\\
    \$ & 4 in 5 & Used in business\\
    $\Psi^2_1$ & 1 in 40,000& Unexplained usage\\
  \bottomrule
\end{tabular}
\end{table}

To set a wider table, which takes up the whole width of the page's live area, use the environment \textbf{table*} to enclose the table's contents and the table caption.  As with a single-column table, this wide table will ``float'' to a location deemed more desirable. Immediately following this sentence is the point at which Table~\ref{tab:commands} is included in the input file; again, it is instructive to compare the placement of the table here with the table in the printed output of this document.

\begin{table*}
  \caption{Some Typical Commands}
  \label{tab:commands}
  \begin{tabular}{ccl}
    \toprule
    Command &A Number & Comments\\
    \midrule
    \texttt{{\char'134}author} & 100& Author \\
    \texttt{{\char'134}table}& 300 & For tables\\
    \texttt{{\char'134}table*}& 400& For wider tables\\
    \bottomrule
  \end{tabular}
\end{table*}

\section{Math Equations}
You may want to display math equations in three distinct styles:
inline, numbered or non-numbered display.  Each of
the three are discussed in the next sections.

\subsection{Inline (In-text) Equations}
A formula that appears in the running text is called an
inline or in-text formula.  It is produced by the
\textbf{math} environment, which can be
invoked with the usual \texttt{{\char'134}begin\,\ldots{\char'134}end}
construction or with the short form \texttt{\$\,\ldots\$}. You
can use any of the symbols and structures,
from $\alpha$ to $\omega$, available in
\LaTeX~\cite{Lamport:LaTeX}; this section will simply show a
few examples of in-text equations in context. Notice how
this equation:
\begin{math}
  \lim_{n\rightarrow \infty}x=0
\end{math},
set here in in-line math style, looks slightly different when
set in display style.  (See next section).

\subsection{Display Equations}
A numbered display equation---one set off by vertical space from the
text and centered horizontally---is produced by the \textbf{equation}
environment. An unnumbered display equation is produced by the
\textbf{displaymath} environment.

Again, in either environment, you can use any of the symbols
and structures available in \LaTeX\@; this section will just
give a couple of examples of display equations in context.
First, consider the equation, shown as an inline equation above:
\begin{equation}
  \lim_{n\rightarrow \infty}x=0
\end{equation}
Notice how it is formatted somewhat differently in
the \textbf{displaymath}
environment.  Now, we'll enter an unnumbered equation:
\begin{displaymath}
  \sum_{i=0}^{\infty} x + 1
\end{displaymath}
and follow it with another numbered equation:
\begin{equation}
  \sum_{i=0}^{\infty}x_i=\int_{0}^{\pi+2} f
\end{equation}
just to demonstrate \LaTeX's able handling of numbering.

\section{Figures}

The ``\verb|figure|'' environment should be used for figures. One or more images can be placed within a figure. If your figure contains third-party material, you must clearly identify it as such, as shown in the example below.

\begin{figure}[h]
  \centering
%%%  \includegraphics[width=\linewidth]{Figures/sample-franklin.png}
  \caption{1907 Franklin Model D roadster. Photograph by Harris \& Ewing, Inc. [Public domain], via Wikimedia Commons. (\url{https://goo.gl/VLCRBB}).}
  \Description{The 1907 Franklin Model D roadster.}
\end{figure}

Your figures should contain a caption which describes the figure to the reader. Figure captions go below the figure. Your figures should {\bf also} include a description suitable for screen readers, to assist the visually-challenged to better understand your work.

Figure captions are placed {\it below} the figure.

\subsection{The ``Teaser Figure''}

A ``teaser figure'' is an image, or set of images in one figure, that are placed after all author and affiliation information, and before the body of the article, spanning the page. If you wish to have such a figure in your article, place the command immediately before the \verb|\maketitle| command:
\begin{verbatim}
  \begin{teaserfigure}
    \includegraphics[width=\textwidth]{sampleteaser}
    \caption{figure caption}
    \Description{figure description}
  \end{teaserfigure}
\end{verbatim}

\section{Citations and Bibliographies}

The use of \BibTeX\ for the preparation and formatting of one's references is strongly recommended. Authors' names should be complete --- use full first names (``Donald E. Knuth'') not initials (``D. E. Knuth'') --- and the salient identifying features of a reference should be included: title, year, volume, number, pages, article DOI, etc. 

The bibliography is included in your source document with these two commands, placed just before the \verb|\end{document}| command:
\begin{verbatim}
  \bibliographystyle{ACM-Reference-Format}
  \bibliography{bibfile}
\end{verbatim}
where ``\verb|bibfile|'' is the name, without the ``\verb|.bib|'' suffix, of the \BibTeX\ file.

Citations and references are numbered by default. A small number of ACM publications have citations and references formatted in the ``author year'' style; for these exceptions, please include this command in the {\bf preamble} (before ``\verb|\begin{document}|'') of your \LaTeX\ source: 
\begin{verbatim}
  \citestyle{acmauthoryear}
\end{verbatim}

Some examples.  A paginated journal article \cite{Abril07}, an enumerated journal article \cite{Cohen07}, a reference to an entire issue \cite{JCohen96}, a monograph (whole book) \cite{Kosiur01}, a monograph/whole book in a series (see 2a in spec. document)
\cite{Harel79}, a divisible-book such as an anthology or compilation \cite{Editor00} followed by the same example, however we only output the series if the volume number is given \cite{Editor00a} (so Editor00a's series should NOT be present since it has no vol. no.),
a chapter in a divisible book \cite{Spector90}, a chapter in a divisible book in a series \cite{Douglass98}, a multi-volume work as book \cite{Knuth97}, an article in a proceedings (of a conference, symposium, workshop for example) (paginated proceedings article) \cite{Andler79}, a proceedings article with all possible elements \cite{Smith10}, an example of an enumerated proceedings article \cite{VanGundy07}, an informally published work \cite{Harel78}, a doctoral dissertation \cite{Clarkson85}, a master's thesis: \cite{anisi03}, an online document / world wide web resource \cite{Thornburg01, Ablamowicz07, Poker06}, a video game (Case 1) \cite{Obama08} and (Case 2) \cite{Novak03} and \cite{Lee05} and (Case 3) a patent \cite{JoeScientist001}, work accepted for publication \cite{rous08}, 'YYYYb'-test for prolific author \cite{SaeediMEJ10} and \cite{SaeediJETC10}. Other cites might contain 'duplicate' DOI and URLs (some SIAM articles) \cite{Kirschmer:2010:AEI:1958016.1958018}. Boris / Barbara Beeton: multi-volume works as books \cite{MR781536} and \cite{MR781537}. A couple of citations with DOIs: \cite{2004:ITE:1009386.1010128,Kirschmer:2010:AEI:1958016.1958018}. Online citations: \cite{TUGInstmem, Thornburg01, CTANacmart}.

\section{Acknowledgments}

Identification of funding sources and other support, and thanks to individuals and groups that assisted in the research and the preparation of the work should be included in an acknowledgment section, which is placed just before the reference section in your document. 

This section has a special environment:
\begin{verbatim}
  \begin{acks}
  ...
  \end{acks}
\end{verbatim}
so that the information contained therein can be more easily collected during the article metadata extraction phase, and to ensure consistency in the spelling of the section heading. 

Authors should not prepare this section as a numbered or unnumbered {\verb|\section|}; please use the ``{\verb|acks|}'' environment.

\section{Appendices}

If your work needs an appendix, add it before the ``\verb|\end{document}|'' command at the conclusion of your source document. 

Start the appendix with the ``\verb|appendix|'' command:
\begin{verbatim}
  \appendix
\end{verbatim}
and note that in the appendix, sections are lettered, not numbered. This document has two appendices, demonstrating the section and subsection identification method.

\section{SIGCHI Extended Abstracts}

The ``\verb|sigchi-a|'' template style (available only in \LaTeX\ and not in Word) produces a landscape-orientation formatted article, with a wide left margin. Three environments are available for use with the ``\verb|sigchi-a|'' template style, and produce formatted output in the margin:
\begin{itemize}
\item {\verb|sidebar|}:  Place formatted text in the margin.
\item {\verb|marginfigure|}: Place a figure in the margin.
\item {\verb|margintable|}: Place a table in the margin.
\end{itemize}

%
% The acknowledgments section is defined using the "acks" environment (and NOT an unnumbered section). This ensures
% the proper identification of the section in the article metadata, and the consistent spelling of the heading.
\begin{acks}
To Robert, for the bagels and explaining CMYK and color spaces.
\end{acks}

%
% The next two lines define the bibliography style to be used, and the bibliography file.
\bibliographystyle{ACM-Reference-Format}
\bibliography{splc19}

% 
% If your work has an appendix, this is the place to put it.
\appendix

\section{Research Methods}

\subsection{Part One}

Lorem ipsum dolor sit amet, consectetur adipiscing elit. Morbi malesuada, quam in pulvinar varius, metus nunc fermentum urna, id sollicitudin purus odio sit amet enim. Aliquam ullamcorper eu ipsum vel mollis. Curabitur quis dictum nisl. Phasellus vel semper risus, et lacinia dolor. Integer ultricies commodo sem nec semper. 

\subsection{Part Two}

Etiam commodo feugiat nisl pulvinar pellentesque. Etiam auctor sodales ligula, non varius nibh pulvinar semper. Suspendisse nec lectus non ipsum convallis congue hendrerit vitae sapien. Donec at laoreet eros. Vivamus non purus placerat, scelerisque diam eu, cursus ante. Etiam aliquam tortor auctor efficitur mattis. 

\section{Online Resources}

Nam id fermentum dui. Suspendisse sagittis tortor a nulla mollis, in pulvinar ex pretium. Sed interdum orci quis metus euismod, et sagittis enim maximus. Vestibulum gravida massa ut felis suscipit congue. Quisque mattis elit a risus ultrices commodo venenatis eget dui. Etiam sagittis eleifend elementum. 

Nam interdum magna at lectus dignissim, ac dignissim lorem rhoncus. Maecenas eu arcu ac neque placerat aliquam. Nunc pulvinar massa et mattis lacinia.

\end{document}

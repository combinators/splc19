variants are the SPL frameworks supported by the most extensive commercial case-studies in our list. A feature they
share with CLS is support for domain spaces with non-boolean types (e.g., numbers), which can also be automatically
computed. This exceeds the capabilities of feature diagrams, where extensions have been proposed to include numbers for
modeling cardinalities .\cite{Sousa:2016:EFM:2934466.2934475}. Inclusion of non-booleans blows up
 the order of magnitude of possible products from 2n , where n is the number of decisions to be made, to infinity.

The authors of ~\cite{Apel:2009:FLA:1555001.1555038} explains the importance of supporting multiple target languages and one of the main improvements of
their FeatureHouse framework over AHEAD ~\cite{Batory2004FeatureorientedPA} is to eliminate language extension overhead. From a BNF grammar with some
annotations, FeatureHouse generates a parser and pretty printer for superimposed AST components. Given an ordered
sequence of these components, they can be automatically composed to generate an AST, which is then pretty printed
to create a target source code artifact. CLS is in some sense complementary to this approach: it automates finding the
sequence in which AST manipulations are to be made and leaves their specification to the developer. In principle, the
two frameworks could be combined. This seems especially promising for the generation of parsers and pretty printers,
which is currently one of the major efforts when adapting CLS to a new target language.

The DOPLER ~\cite{Dhungana:2011:DMD:1924082.1924092} framework avoids any assumptions about target languages and instead
 operates with a decision-based meta model of code generators. Code generators are automatically executed by the
 rule-based Drools engine ~\cite{DBLP:conf/agtive/2007} whenever decisions are taken. This offers similar flexibility to combinators in CLS,
 which can perform arbitrary computation. A downside of the DOPLER framework is that the low level, solution space
 specific collaboration of code generators has to be coded manually. In CLS, code generating combinators can specify
 their need for input from other code generators in their type, and the inhabitation algorithm will automatically
 compose them. In DOPLER this solution space dependency would have to be hard coded or pushed to the domain space
 to be handled by Drools.


In the broader scope of software product lines beyond a specific framework, a feature-based extension to UML
is proposed in ~\cite{Czarnecki:2006:VFM:1173706.1173738}. In contrast to our modeling of the domain specific
variability with classes, this extension modifies models of solution domain classes depending on Object Constraint
Language (OCL) expressions.

